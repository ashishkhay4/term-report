\documentclass[12pt]{article}

\usepackage{graphicx}
\graphicspath{{ASHISH}}

\begin{document}
\begin{figure}
\section*{PROJECT REPORT \\On\\\\Title : PORTABLE VENTILATOR IN HEALTHCARE\\\\Submitted by : ASHISH \\Roll.no :21111013\\Details :1st Semester ,BioMedical Engineering \\\\National Institute Of Technology ,Raipur}


\centering
\includegraphics[scale=0.6]{collage.png}

\section*{Supervision of : Saurabh Gupta Sir\\Professor\\BioMedical Department }



\end{figure}

\clearpage
\section*{Acknowledgement}
I am grateful to Saurabh Gupta Sir ,BioMedical Department for this proficient supervision of the term project on "Portable Ventilator In Healthcare". I am very thankful to you sir for your guidance and support.



\section*{Name : Ashish\\Roll.no : 21111013\\Details : 1st semester , BioMedical Engineering\\National Institute Of Technology ,Raipur}

\clearpage
\section*{"PORTABLE VENTILATOR IN Healthcare "}
\section*{ASHISH and STUDENT OF NIT RAIPUR}
\section*{1 . Table of Contents:} 


\section*{Section :}
Abstract\\\\Background\\\\Product/Project Description\\\\Product Research\\\\Technology Research\\\\Marker Research\\\\Market Description\\\\Marketing Requirement\\\\Block Diagram\\\\Conclusion\\\\References

\clearpage
\section*{Abstact : }
The current COVID-19 pandemic has heavily impacted the healthcare system in the
United States and elsewhere. The need for patients to have access to a hospital with a ventilator
along with a shortage of ventilators for recovery and at-home care as a result of minimal hospital
vacancy for patients has been greatly stressed. The presented problem is both an unmet demand
and supply of portable and effective ventilators.


The solution is to design a ventilator which can meet the symptoms and strains which
COVID-19 can put on various individuals, should they not have access to a commercial
ventilator as an economic constraint, or have restricted access to a medical facility attributed to
the influx of patients. This means that a portable ventilator targeted for lower risks patients to be
the ideal ventilator design. This ventilator would allow for short term out-patient care of low risk
patients, thus allowing hospitals to focus on high risk patients.


 The ventilator would
include wireless communication, smartphone integration, and a portable power system which
would allow consumers to relocate the ventilator as needed and have back-up power in the event
of a power outage.

\section*{Background : }
A ventilator is a machine that uses mechanical ventilation to help patients breathe and
provide oxygen when their own body is not able to do this correctly. The first widely used
ventilators were during the polio epidemics,1920s and 1930s, and called iron lungs. These were
large noninvasive devices that required most of the patient’s body to be in a box. Then it used
negative pressure to expand/contract the patient’s chest and thus also the lungs, drawing air.

\centering
\includegraphics[scale=0.4]{lungs.png}


In the following years, the ventilators were constantly improved to function better,
minimize maintenance, and increase reliability. However, the next big revolution in ventilator
design did not occur until the introduction of microprocessors. This third generation of
ventilators started with the Drager EV-A and finally allowed for monitoring of patients on a
3
screen. Immediately following, were the Puritan Bennett 7200, the Bear 1000, the Servo 300,
and the Hamilton Veolar models. 


\includegraphics[scale=0.7]{helpp.jpg}
\clearpage
\section*{Product/Project Description : }
The Portable Ventilator is a mechanical ventilation device designed to tackle one of the
biggest problems with current ventilators, their restrictiveness. Currently patients are required to
either be hospitalized and occupy an ICU room risking contracting disease or forgoing care. This
leaves many patients with milder symptoms to have two bad options. Along with this, hospitals
are also at a loss. Every patient with milder symptoms that is hospitalized is one less ICU room
available for other patients with more severe symptoms. This also puts a strain on the limited
resources hospitals have, such as nurses.


\includegraphics[scale=0.7]{ash.jpg}

The Portable Ventilator seeks to solve these problems by providing a third option to
patients and hospitals. This is the ability to provide out-patient care for mild cases that require
help breathing and oxygen intake. This would be done by designing the Portable Ventilator to be
lighter, smartphone compatible, automatic monitoring and adjustment, and a simpler interface for
controlling the device. This would allow the ventilator to be used at the patients home with no
need for professional monitoring. This would meet both the patient’s need to be provided care
without hospitalization and the hospital’s need to keep available rooms and staff.

\clearpage
\section*{Product Research : }
Today most of the ventilators in the market and in use are designed for hospital use and to
address specific medical needs. This means that an ICU ventilator may not be able to be used for
neonatal ventilation. Having a many different types of ventilators allows most patients to be
treated for whatever condition they may get. Modern ventilators include many health safety
features to ensure that the patient remains safe even when there is an error.


 This includes
alarms for dangerous changes in patients’ biometrics. Another safety feature is the monitoring
and displaying of both patient and device conditions. One of the most important features of
current ventilators is the ability to change the settings and provide precise control of the device.
However, this means that to operate a ventilator not only requires accurate ventilator knowledge
but also accurate medical knowledge. This causes ventilators to require professional monitoring
and administering.

\section*{Technology Research : }
Ventilators are controlled by microprocessors. These microprocessors are responsible for
controlling and communicating with various components. These components include the sensors,
compressors, pumps, displays, and etc. For the Portable Ventilator design we would need a
microprocessor that can perform those same duties using I2C communication protocol along
with having wireless capability. A Raspberry Pi 4 microprocessor is capable . The
sensors needed are many, with flow rate, temperature, humidity, and O2 being a few. There are
many available sensors that could be used which are both small and require little power.

\includegraphics[scale=0.9]{ashi.jpg}


Focusing on mild symptom/low risk patients could allow the Portable Ventilator to be
designed without an air compressor and reduce the cost and complexity of it. Due to the
widespread use of advanced smartphones, the Portable Ventilator can use an app to easily
communicate with the user and allow the user to easily adjust the ventilator.


The most important feature of modern ventilators are the alarms and failsafe features to
protect the patient. This is done through the monitoring of the patient’s biometrics and ventilator
conditions. Any problem encountered by the ventilators would cause the alarms to set off and
failsafe features to activate, if needed, until a nurse can attend to the patient[9]. With the use of
an app, the portable ventilator can guide the patient or family member on the steps needed when
there is a problem and automatically call for medical attention.

\section*{Market Research : }
The ventilator market is a growing market due to both an increase in the geriatric
population and the COVID-19 pandemic.[10] The COVID-19 pandemic rapidly increased the
demand for ventilators and companies responded by increasing production of ventilators.
However, this increase in production was for ICU ventilators. This led to a saturation in the
market as the number of ventilators that can be used is limited by the number of ICU rooms
available. Therefore, the largest reason for patients not receiving the necessary ventilation they
need is not caused by a lack of ventilators but a lack of hospital space and medical staff. This
leaves a need for hospitals to alleviate their in-patient care and increase out-patient care. The
Portable Ventilator is designed to meet that need and fill that market.


Along with opening up hospital space, the Portable Ventilator will also fulfill the need for
patients to remain active and comfortable. Presenting this new option should cause many current
patients to opt into using the Portable Ventilator over the current one they are using.
While the current demand is being driven mainly by the current pandemic, the demand
will continue to increase for years due to the geriatric population. The limitations for this
increase are due to high cost and operating requirements . The Portable Ventilator would
address these issues by presenting a cheaper, lighter, and accessible alternative for out-patient
care in both nursing and private homes.

\section*{Market Description : }
The portable ventilator aims to create an avenue for hospitals to provide out-patient care
so that they may increase the amount of rooms available for patients with more severe
respiratory symptoms while still administering adequate out-patient care for those with
less-severe symptoms. The ventilator will be driven with a desire to make sure all patients with
respiratory symptoms may receive the care they need appropriate to their symptom severity. This
is much inspired by the 2019 outbreak of COVID-19 and its impact on hospitals with a focus on
the United States.


To penetrate the market, serious capital would have to be invested into getting this device
FDA approved, patent the device, visit trade shows and hospitals to talk with hospital business
administrators, and designing and building a working prototype. This will require a minimum of
25,000 of funding to provide for the materials to produce a ventilator unit, for the labor of the
team, for the applications towards patents and FDA approval, and for travel to trade shows and
hospitals. The time investment would be large for the team involved. Due to the small team, it
could take around 20 hours per person in order to get a unit functional. It also may take two years
to get all of the approvals and begin to enter the market.

\clearpage
\section*{Marketing Requirements : }
The Portable Ventilator will be designed to include features that will solve the current
problem with available ventilators. Similar to current ventilators, the Portable Ventilator will
provide breathing and oxygen intake assistance to low risk patients. The differentiator between
current ventilators and the Portable Ventilator will be the greater accessibility, mobility, and
16
comfort the Portable Ventilator will provide to patients. This will be done by designing the
ventilator to be light, run on a battery, and be easy to use.


The Portable Ventilator places higher importance on features that add to the functionality
of the device. If there is a choice between a higher cost or a lower functionality of a smartphone
integration, the ventilator will have to receive the higher cost in order to provide the extra service
to the customer. This is necessary to stand out versus other portable ventilators on the market.
Without many of these beneficial design features, this device would not be able to break through
the market.

\section*{Block Diagram : }
The high(0th) level block diagram for the ventilator displays the necessary inputs and
outputs for the system. The inputs for the system are the patient’s biometrics, patient input, air,
and a charger. The patient’s biometrics must be accurately measured to provide feedback to the
ventilator so it can adjust to provide the needed enriched air. The patient input sets and changes
the settings for the ventilator and defines how the ventilator will be used(e.g. high flow rate). The
air input is used by the system to provide the air flow rate and is enriched with oxygen to create
the enriched air for the patient. The charger input is necessary to charge the ventilator’s battery
and allow the ventilator to be portable. The outputs for the system are the enriched air and the
measured biometrics. The enriched air is humid, pressurized, oxygen-enriched air that is supplied
to the patient. The enriched air’s conditions must be carefully and accurately met to provide the
patients needs. The measured biometrics must be clearly and readily available to the patient.

\includegraphics[scale=1.3]{asp.png}
This will require both a functional and usability test. The functional test will be a check
that the phone/app is correctly communicating to the Raspberry Pi and the phone/app is correctly
displaying information. The usability test is to ensure that the ventilator’s setting can be easily
set and managed. This test will be done by having people outside the project use the app to set
the ventilator’s setting according to what they are told they should be.


 A notable thing about the block diagram is that the Lion Battery and
Voltage regulator will connect and provide the power to the various components. The feedback
for the system is provided through the combination of the sensors and the Raspberry Pi. The
pulse oximeter and watch both measure the patient’s biometrics with the pulse oximeter being
the back up.

\clearpage
\section*{Conclusion : }
Various portable ventilators are available, driven by demands for equipment suitable for different clinical situations and environments. Their design reflects availability of gas and electrical supplies and the modes of ventilatory support required by the patient population. When transporting critical care patients, provision for estimated gas and electrical requirements should be made. A ventilatory mode suited to the patient’s clinical condition should be selected and trialled before departure.


An appreciation of how different portable ventilators function, preferably supported by comparative data, can help when an organization purchases such equipment. Understanding the strengths and weaknesses of a specific ventilator and breathing circuit may help anticipate and prevent complications during transfer.


A portable ventilator should be lightweight, robust, and able to function in demanding environments with little maintenance.

Most portable ventilators display the oxygen concentrations selected by the operator, and do not measure that delivered by the ventilator.

Portable ventilators may not provide identical support to the ICU machine in use despite apparently similar settings; a trial period should always be allowed before moving the patient.

Portable ventilators that assist spontaneously breathing patients are more complex and generally require microprocessor control.

Alternatives to oxygen cylinders such as oxygen concentrators and liquid oxygen should be considered for prolonged use outside of hospital.


\clearpage
\section*{References : }

1.   	SIMS pneuPAC, VentiPAC 200D Ventilator User’s Manual, 2001LutonSIMS pneuPAC Ltd.



2.   	Dräger Medical, Oxylog 3000 Emergency and Transport Ventilator—Instructions for Use, 20022nd Edn.Hemel HempsteadDräger Medical UK.


3.   	Pulmonetic Systems, Instruction Manual for the LTV 1000 Transport Ventilator., 1999Colton, CAPulmonetic Systems Inc.


4.   	McCluskey A,  Gwinnutt CL. Evaluation of the Pneupac Ventipac portable ventilator: comparison of performance in a mechanical lung and anaesthetized patients, Br J Anaesth, 1995, vol. 75 (pg. 645-50).


5.   	Campbell RS,  Johanningman JA,  Branson RD,  Austin PN,  Matacia G,  Banks G. Battery duration of portable ventilators: effects of control variable, positive end-expiratory pressure, and inspired oxygen concentration, Respir Care, 2002, vol. 47 (pg. 1173-83).


6.   	Austin PN,  Campbell RS,  Johannigman JA,  Branson RD. Work of breathing characteristics of seven portable ventilators, Resuscitation, 2001, vol. 49 (pg. 159-67).


7.   	Emergency Care Research InstitutePortable/Transport Ventilators, Health Devices, 2004, vol. 33 (pg. 381-401).


8.   	Martin T. , Aeromedical Transportation—A Clinical Guide., 2006Aldershot, HampshireAshgate Publishing Limited.


9.   	Lowes T,  Sharley P. Oxygen Conservation during long-distance transport of ventilated patients, Air Med J, 2005, vol. 24 (pg. 164-71).


10.  	Simonds AK. Home ventilation, Eur Respir J, 2003, vol. 22 (pg. 38S-46S).

\end{document}